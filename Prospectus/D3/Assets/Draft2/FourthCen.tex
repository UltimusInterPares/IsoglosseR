\section{4\textsuperscript{th} cen.  BCE}

Three cities represent the fourth century \textsc{bce}.
Megara provides nine inscriptions;
Tanagra five;
and Oropus eight,
for a total of twenty-two data.
Bœotian accounts for fifty-nine percent of the inscriptions ($\frac{13}{22}=.59$),
and Doric for forty-one percent ($\frac{9}{22}=.41$).
However,
four of the inscriptions from Oropus come from matched sets.
EO~385 is equivalent to \ig{266} and \ig{267} combined,
and EO~386 is equivalent to \ig{236} and \ig{238}.
As these are listed as four separate inscriptions in the IG,
they will be treated as such here.


\subsection{4\textsuperscript{th} cen.  Megara}
The inscriptional corpus readily attests that Megarians retained their long alphas.
The inscriptions \ig{1 - 7},
all very formulaic honorary inscriptions, include the names \textel{Δ\A{ᾱ}μεα Δᾱμοτέλης}
and \textel{Δᾱμοτέλης Δ\A{ᾱ}μεα}.
The name \textel{Δ\A{ᾱ}μεα} is obscure,
but \textel{Δᾱμοτέλης} is easily analyzed as being built from the root \groot{δᾱμ}.

This root is also attested in the singular declensions of δᾶμος. 
\ig{1} and \ig{2} both attest the nominative \textel{δᾶμος},
\ig{1 - 6} show the genitive \textel{δ\A{ᾱ}μου} and dative \textel{δ\A{ᾱ}μωι}.
As transcribed on the PHI website,
the inscriptions disagree on how to accent the accusative form.
\ig{5} shows a paroxytone \textel{δ\A{ᾱ}μον}
while \ig{4} and \ig{6} show a propersipomenon \textel{δᾶμον}.
This discrepancy may be due to editorial disagreements among the editing staff
over how the \textel{σωτῆρα} rule applies or does not apply to Doric Greek. 
Per Philomen Probert, 
this rule states that the ``accent on a long vowel in a penultimate syllable must be a circumflex
if the vowel of the final syllable is short.''\autocite[61]{Probert}
Also, per Probert, 
``the \textel{σωτῆρα} rule did not apply in Doric,''\autocite[71]{Probert}
hence the decision by the transcriber of \ig{5} to produce \textel{δάμον}.
The two forms should be treated as equivalent: both point to the underlying root \groot{δᾱμ}.

The inscriptions \ig{1},
\ig{5},
and \ig{6} have some form of the name \textel{Δᾱμ\A{ᾱ}τριος} -- 
counterpart to Attic \textel{Δημήτριος} whence Demetrius.
The origin of the component \textel{Δᾱ-} is contested,
but the component \textel{-μ\A{ᾱ}τρ-} is from the root \groot{μ\A{ᾱ}τ(ε)ρ},
found in Attic \textel{μήτηρ}.
So we would expect Megarian \textel{μ\A{ᾱ}τηρ},
which is what we find in \ig{55},
a funerary monument for the children \textel{Πύθων} and \textel{Λυσικράτεια Ἀναξίωνος}
who graciously retired from this mortal coil
in order that their mother may leave behind the phrase \textel{\Rm{α} μ\A{ᾱ}τηρ ἀνέθηκεν},
confirming the Megarian Doric root \groot{μᾱτ(ε)ρ}.

Similar to \textel{δᾶμος},
the singular declension of \textel{βουλ\A{ᾱ}}
is attested in the sample.
The inscription \ig{5} shows the genitive \textel{βουλᾶς} while inscription \ig{6} shows the dative \textel{βουλᾶι}.
Inscriptions \ig{1 - 4} and \ig{7} show both.
A partial declension is also attested for \textel{γᾶ}: \ig{2} and \ig{3} both show the dative \textel{γᾶι},
while \ig{2 - 7} show \textel{γᾶγ},
which is a form of the accusative \textel{γᾶν} used when the following word begins with a velar,
as in \ig{4} \textel{κατὰ γᾶγ καὶ κατὰ θάλασσαν}.

Finally,
the Megarian inscriptions partially attest the singular declension of the feminine article \textel{\Rm{α}}.
The nominative is present in \ig{1 - 3} in the phrase \textel{\Rm{α} πόλις},
as well as in \ig{55} \textel{\Rm{α} μάτηρ} from above.
The genitive is in \ig{1 - 7} as part of the phrase \textel{τᾶς πόλιος},
and the dative as \textel{τᾶι βουλαῖ}.
The singular feminine accusative article is not attested in the sample from the fourth century,
but we can assume the form \textel{*τ\A{ᾱ}ν} based on the precedent set by \textel{δᾶμος} and \textel{βουλά}.

The Megarian inscriptions firmly assert that local speakers retained their long alphas in all positions.
The roots \groot{δᾱμ} and \groot{μᾱτ(ε)ρ} show long alphas in some basic roots.
The forms of \textel{γᾶ} have the exciting effect of showing that
the Attic-Ionic Vowel Shift must have occurred sometime after the earliest vowel contractions.
\textel{Γᾶ} does not fit the conditions for the \textel{σωτῆρα} rule,
so the circumflex must necessarily be the result of a vowel contraction.
However,
\textel{γῆ} must have been the result of the Vowel Shift
since the only combination of short vowels that would contract in Attic to \textel{η} and in Doric to \textel{ᾱ} is the segment \textel{ᾱε}. 
Given that \textel{γῆ} declines as a typical first declension noun,
then the second vowel in the contracted segment must have been an *a,
eliminating the pre-contraction form \iform[L]{g\={a}e} as an option.
The only way to show a Doric \textel{γᾶ} and Attic \textel{γῆ} would be for some prior form *gVV
to contract to *gā,
which was inherited directly in Doric and which later shifted to \textel{γῆ} in Attic.

The forms of \textel{βουλ\A{ᾱ}} and the article \textel{\Rm{α}} show long alphas retained in case endings,
which suggest further retention in other affixes as well.
With all this in mind,
Megara is categorized as being on the \textel{δᾶμος} side of the isogloss line.
The line then must be drawn starting to the city's east,
somewhere between its borders and those of Eleusis.

\subsection{4\textsuperscript{th} cen.  Tanagra}
The sample from Tanagra, 
though fragmentary,
tells a similar story to Megara's.
Here return the patronymics in \textel{-δᾱς}.
\ig{537} contains the names \textel{Ξενοκλίδᾱς Ἀνδρομάχιος},
\textel{Τυρμίδᾱς Μεγαλίνιος},
and \textel{Βρομίδᾱς Θαρσυκλεῖος};
as well as the fragmentary names \textel{\rbrk λιάδας Κοιρατάδᾱο} and \textel{\dots άδᾱς Εὐφαμίδᾱο}.
The singular genitive \textel{-δᾱο} is also attested in these last two,
as well as in the fragment \textel{\dots οτέλεις Κλεαρχίδᾱο},
and the names \textel{Θιογίτων Καλλικλίδᾱο} and \textel{Ξενότιμος Χαρώνδᾱο}.

\ig{538} shows a similar fascination with patronymic systems.\footnote{Or am I projecting?}
The nominative is attested in such forms as \textel{Δᾱλιάδᾱς Νικολά\ladd{ιος}},
\textel{Θρασυκλίδᾱς Ἐχετίμιος},
\textel{Ὀνατορίδᾱς Μνᾱσιόργος},
\textel{Πιστίδᾱς Ἀντιγώνιος},
and \textel{Εὐκλίδᾱς Σιμμιῆος}.
The genitive is present in two names: \textel{Κόμων Πατροκλίδᾱο} and \textel{Τιμήνετος Προκλίδᾱο}. 
\ig{537} and \ig{538} both have one instance of the root \groot{δᾱλ}, each in personal names.
The former, 
\textel{Δᾱλιόδωρος}; and the latter, 
\textel{Δᾱλιάδᾱς Νικολά[ιος]} as above.
There is also a single instance of the suffix \textel{-νᾱς} in the sample:
\textel{537} shows the name \textel{Δωρίνᾱς Ἱαροτέλειος}.

As a result of their relative scarcity and fragmentary nature,
the data from Tanagra are less denotative overall than those from Megara.
That being said, the information therein nevertheless suggests that Tanagrans retained the long alpha in all positions.
One important counterexample is the proper name \textel{Εὐβουλίδης} found in \ig{552}.
However,
this is the same inscription that provides the term \textel{μνᾱμεῖον},
and even \textel{νικ\A{ᾱ}σαντος},
the singular genitive participle of \textel{νικάω}
which corresponds with the Attic-Ionic form \textel{νικήσαντος}.
This inscription may have been commissioned by a foreigner,
as with \ig{533},
whose dedicant clearly labels himself \textel{Λυσίστρατος Θηβαῖος} --
Lysistratus the Theban.
A foreigner writing for the local audience might understandably have their name engraved as they pronounce it
while composing the rest of their inscription in the majority dialect.
In any case, 
the name alone is not enough to negate the rest of the Tanagran evidence,
and so Tanagra should be categorized as on the \textel{δᾶμος} side of the isogloss line.
The line should then start between Megara and Eleusis,
where it moves north between first Mt. Parnes (modern-day \textel{Πάρνηθα}/Parnitha)
and Mt. Kerata (modern-day \textel{Πατέρας}/Pateras),
then between Mt. Parnes and Mt. Kithæron.
The line then turns east, passing between Mt. Parnes and Tanagra.

\subsection{4\textsuperscript{th} cen.  Oropus}
The isogloss being so defined by Mt. Parnes, 
we expect Oropus to be categorized with Megara and Tanagra.
The inscriptional evidence, however,
contradicts this expectation.
As early as the fourth century \textel{bce}, 
Oropus had abandoned most of its long alphas in favor of etas.

The inscriptions from this century show one instance of a patronymic in \textel{-δᾱς},
but with an eta in place of the alpha: \textel{Μεγακλείδην} in \ig{327}.
Meanwhile,
\textel{235} shows four instances of the root \groot{δᾱμ}, but again with an eta. 
These are \textel{δημότης} and \textel{δημοτέων} ``commoner,''
and two instances of \textel{δημορίων} ``public,''
counterpart to Attic \textel{δημοσίων}.

In keeping with this shift,
\ig{235} shows \textel{τοῖ κοιμητηρίοι} and \textel{τὸ κοιμητήριον} ``cemetary.''
These are built from the verb \textel{κοιμάω} ``sleep,  perish,''
so we would expect the stem to retain its alpha in a \textel{δᾶμος} dialect,
as in the Doric future \textel{κοιμ\A{ᾱ}σω}.
If Oropus had retained their alphas, then the reflex would have been \iform[G]{κοιμᾱτήριον}. 
Finally,
\ig{235} shows two instances of the root \groot{\Rm{α}μερ},
both as plural accusatives of time.
These are \textel{τρεῖς ἡμέρᾱς} and \textel{δέκα ἡμέρᾱς}.

Overall, the data from Oropus suggest that we should classify Oropus as on the \textel{δᾶμος} side of the isogloss line --
an intriguing break of rank amongst the Bœotians. 
The isogloss line then must pass between Tanagra and Oropus.
It should turn northeast from its position south of Tanagra,
passing between the two cities and tracing a vague border between Bœotia and its subregion Oropia.