\section{5\textsuperscript{th} cen.  BCE}

\subsection{5\textsuperscript{th} cen.  Tanagra}

A single inscription from Tanagra represents the fifth century \textsc{bce}. 
The inscription is published in its entirety as \ig{585}, 
with contextual information published in SEG 19:337.\autocite{SEG} 
It is a stele cut of black stone with four sides, 
each bearing a list of personal names. 
The SEG explains that the names list Tanagrans who fell in the Battle of Delion. 

Note that the Tanagrans in the fifth century used an did not write with the letter eta, 
but wrote every ``e sound'' with an epsilon.
They also wrote the letter heta -- 
which is of the same shape as an eta -- 
to indicate rough breathing. 

The names on the stele suggest that Tanagrans retained a long alpha in places where Attic and Ionic wrote an eta. 
The suffix \textel{-δᾱς},
standing for Attic \textel{-δης},
shows up in fourteen different names. 
These include such hits as \textel{Γοθθίδᾱς},
\textel{Μεγγίδᾱς},
and the incredible \textel{Ηισσταΐδᾱς}.
Similarly, the name \textel{Αἰσκίνᾱς} (perhaps with the suffix \textel{-ι-νᾱ}\autocite[861.11]{Smyth}) 
contrasts with the Attic form \textel{Αἰσκίνης}.

Moving onto whole roots, we can disassemble most of the names into their constituent parts. 
Five of these are built with the root \groot{δᾱμ}, 
the counterpart to \groot{δημ} in the Attic \textel{δῆμος}.
These names are \textel{Δᾱμότιμος},
\textel{Φᾱνάδαμος},
\textel{Ἀρισστόδᾱμος},
\textel{Δᾱμομέλ\M{ο}ν},
and the intriguing \textel{Δᾱμόξενος}.
One name is built from the root \groot{μνᾱ}, counterpart to \groot{μνη} in the Attic \textel{μνήμη}:
that is the exceptionally optimistic \textel{Ἀριόμνᾱτος}.
Another both is built from the root \groot{δᾱλ} (Attic \textel{Δῆλος})
and uses the very Homeric patronymic suffix \textel{-ιάδᾱς}: that is \textel{Δᾱλιάδᾱς}.

All of this indicates that Tanagrans in the fifth century \textsc{bce} did not participate in any changes to their long alphas in either roots or affixes. 
Unfortunately, 
one city alone does not provide enough evidence to draw a meaningful isogloss;
however,
the data it provides makes for a practical blueprint when examining later evidence. 
It tells us to be on the lookout for any long alphas, 
be they from a root or affix, 
and provides a small sample of valuable segments that will be used as a lens through which we examine the later evidence. 
These are any words built off of the roots \groot{δᾱμ},
\groot{μνᾱ},
and \groot{δᾱλ};
and any words suffixed with \textel{-νᾱς} or the patronymic \textel{-δᾱς}.
