\section{3\textsuperscript{rd} cen.  BCE}

The largest inscriptional sample represents the third century \textsc{bce}.
There are twenty-five from Megaris:
six from Megara proper,
two from Pagæ,
and seventeen from Ægosthena.
This century marks the fashionably late entrance of Pagæ and Ægosthena into the discussion.
Then there are an incredible ninety-two inscriptions from Bœotia.
Eighty-two are from Oropus,
and twelve are from Tanagra.
Altogether,
this makes one hundred and seventeen inscriptions,
making this century the most represented part of the collection.

\subsection{3\textsuperscript{rd} cen.  Pagæ}
Pagæ shows at most two instances of the suffix \textel{-δᾱς}.
\ig{188} attests to the name \textel{Ἀπολλωνίδᾱ},
and \ig{189} shows the fragment \textel{\rbrk ίδᾱν Σατύρου}.
There are only four instances of the root \groot{δᾱμ}.
\ig{188} shows the name \textel{Δᾱμοτίωνα},
as well as the verb form \textel{δᾱμιοργύντων},
which is built from the roots \groot{δᾱμ} and \groot{\eo ργ}.
\ig{189} shows \textel{δᾶμος} in the nominative and \textel{δ\A{ᾱ}μωι} in the dative singular.

Pagæ offers relatively little data,
but the material suggests that speakers there retained the long alpha. 
The presence of both the suffix \textel{-δᾱς}
and the root \groot{δᾱμ} are enough to offer precedence for long alphas appearing in roots and suffixes.
Given the lack of any counterexamples,
Pagæ should be classified as on the \textel{δᾶμος} side of the isogloss line.

\subsection{3\textsuperscript{rd} cen.  Ægosthena}
The Ægosthenian evidence also suggests that,
on average,
local speakers had retained long alphas in regular speech.
\ig{209} shows the root \groot{δᾱμ} as part of the personal name \textel{Χαρίδᾱμος},
and the declensions of \textel{δᾶμος} once again are partially attested.
The nominative plural \textel{τοῖ δ\A{ᾱ}μοι} is present in \ig{207},
the singular dative \textel{τῶι δ\A{ᾱ}μωι} in \ig{208},
\ig{219},
and \ig{213}.
So far,
the plural nominative is novel to us,
but the dative keeps with the Megarian forms seen in the fourth century \textsc{bce}.
This continuity suggests that the declension had not seen many significant changes in the intervening time.

Also in \ig{213} is \textel{γᾶν}, the accusative singular of \textel{γᾶ} (Attic \textel{γῆ}).
Then the Tanagran inscriptions attest a partial declension of the singular feminine article.
The nominative \textel{\Rm{α}} occurs six times in \ig{207}.
The accusative \textel{τ\A{ᾱ}ν} occurs six times in \ig{207},
once in \ig{219},
and twice in \ig{213}.
Finally,
the genitive \textel{τᾶς} occurs in \ig{208},
\ig{219},
and \ig{213}.

The patronymic suffix \textel{-δᾱς} is attested seven times in Tanagra.
\ig{213} has the incredible name \textel{Ἥρωνος τοῦ Ξενοκλείδᾱ τοῦ δὶς Νεοστράτου}.
\ig{217} follows with \textel{Ἐχεκλείδᾱς Εὐπείθου} and the double patronymic \textel{Φυλακίδᾱς Ἀπολλωνίδᾱ}.
\ig{218} has \textel{Διοκλείδᾱς Ξένωνος}, and \ig{220} finishes the suffixes with \textel{Ἄγησος Ἡρίδᾱ}.

The root \groot{μᾱτ(ε)ρ} is attested four times,
each in a declension of the name \textel{Δᾱμ\A{ᾱ}τριος}.
\ig{222} ends with the fragmentary name \textel{Δᾱμ\A{ᾱ}τριος \lbrk},
\ig{212} ends with \textel{Δᾱμ\A{ᾱ}τριος Καλλικλέος},
and right in the middle of \ig{215} is \textel{Σωσικλῆς Δᾱμᾱτρίου}.
The root \groot{μνᾱ},
however,
is only attested once as a component of the name \textel{Μνᾱσάρετος Διοδώρου} in \ig{218}.

The Ægosthenian evidence,
though similarly brief to that of Pagæ,
is enough to indicate that Ægosthena should be categorized as on the \textel{δᾶμος} side of the isogloss line.
The suffix \textel{-δᾱς} is well attested for the city's sample size,
and the repeated use of the name \textel{Δᾱμ\A{ᾱ}τριος} suggest it was a popular name.
Furthermore,
the consistency of the two long alphas implies that the vowel generally kept its quality word-medially.
The positions of Pagæ and Ægosthena should come as no surprise, given their geographical positions relative to one another
and their political connections with Megara.
Both cities sit northwest of Megara on the coast of the Halcyon Bay (modern-day \textel{Κόλπος Αλκυονίδων}/Alkyonides Gulf).
Their location places them far west of the isogloss line,
deep enough into Megaris to deal with more Doric and Bœotian than Attic on a day-to-day basis.

\subsection{3\textsuperscript{rd} cen.  Megara}
The Megarian data also suggests retention of the long alpha in all positions. 
For instance, patronymics in \textel{-δᾱς} are attested nineteen times across five inscriptions.
\ig{42} contains the name \textel{Ἀπολλόδωρα Ἀμφιαρίδᾱ},
\textel{Φίλων Ἀπολλωνίδᾱ},
\textel{Διοκλείδᾱς Πασίωνος},
and the double-hitter \textel{Θηβάδᾱς Νομιάδᾱ}.
\ig{41} has one name,
the alliterative \textel{Δεκάνων Διοκλείδᾱ}.
\ig{29} has the names \textel{Καλλιάδᾱς Μνᾱσιθέου},
\textel{Παριγένης Εὐκλείδᾱ},
\textel{Εὔστραος Ἐρασικλείδᾱ},
\textel{Εὐκλέων Διοκλείδᾱ},
and another \textel{Διοκλείδᾱ} (this one the gymnasiarch). 

Also included are \ig{42}:
\textel{Ἀναξιμένης Σαμώνδᾱ} and \ig{41}:
\textel{Ματρώνδᾱς Λαχάρεος},
but the names are difficult to deconstruct.
We could read the segment \textel{ν-δᾱ} as a patronymic suffix attached to a nasal root,
but nothing resembling \textel{*Ματρων} or \textel{*Σαμων} shows up in the rest of the sample.
Even looking outside of the IG gives very little.
\textel{Ματρώνδᾱς} superficially resembles the root \groot{μᾱτ(ε)ρ},
maybe indicating \textel{μήτρως} ``maternal grandfather,''
giving it the meaning ``ancestor of (my?) maternal grandfather.''
An interesting name to give to a child.
Unfortunately,
the names are ultimately obscure,
and any firm claims about them may be dubious.

\ig{28} has the names \textel{Ἀπολλοδώρου τοῦ Φιλωνίδᾱ},
\textel{Εὐρυκλείδλᾱς Πουλυδ\A{ᾱ}μα},
\textel{Διοκλείδᾱ τοῦ Ἡρογείτου},
and the fragmentary \textel{Ἡρακλείδᾱς Α \ldots ωνος}.
\ig{27} has the final two names: \textel{Κιοκλείδᾱς Χαριδ\A{ᾱ}μου}
and \textel{Πολυδεύκείδᾱς Θεδώρου}.
However, there is only one instance of the suffix \textel{-νᾱς}:
that is the fragmentary name \textel{Καλλίνᾱς \ldots ελλίδᾱ} in \ig{29}.

The root \groot{δᾱμ} occurs seven times in the Megarian inscriptions.
\ig{27} has the names \textel{Δᾱμ\A{ᾱ}τριος Εὐδ\A{ᾱ}μου},
\textel{Χαρίδᾱμος Διοδώρου},
and \textel{Διοκλείδᾱς Χαριδ\A{ᾱ}μου}.
\ig{29} has the name \textel{Ἀρχίλοχος Ἡρόδ\A{ᾱ}μου},
and the singular genitive \textel{τοῦ δ\A{ᾱ}μου} besides.
\ig{41} contains the office title \textel{δᾱμιοργοί} (Attic \textel{δημιουργοί}),
and \ig{42},
the name \textel{Εὔδᾱμος Μᾱτροδώρου}.

The root \groot{μνᾱ} shows up in three personal names.
In \ig{29} are the names \textel{Μνᾱσίθεος Θεδώρου} and \textel{Αὐτόλυκος Μνᾱσιθέου}.
In addition,
\ig{29} contains the name \textel{Καλλιάδᾱς Μνᾱσιθέου}.
The root \groot{μᾱτ(ε)ρ} is similarly represented.
The first instance is in \ig{42} as the name \textel{Διόδωρα Μ\A{ᾱ}τριος};
the second instance,
and final,
is in \ig{27} as the name \textel{Δᾱμ\A{ᾱ}τριος Εὐδ\A{ᾱ}μους}.

Overall,
the Megarian evidence is relatively dense,
offering a few long inscriptions in the place of multiple shorter ones.
The inscriptions assert that Megara should be categorized as on the \textel{δᾶμος} side of the isogloss line.
The inscriptions show long alphas in the same environments as in the fourth century,
occasionally even in the same name.
Our line,
then,
should be drawn starting from the same position as that of the last century: between Megara and Eleusis.

\subsection{3\textsuperscript{rd} cen.  Tanagra}
Also in keeping with the last century,
the inscriptional evidence from Tanagra indicates that the majority of speakers retained their long alphas in all positions.
As for suffixes,
the patronymic \textel{-δᾱς} occurs in eight instances across as many inscriptions.
\ig{504 - 509} respectively have the names \textel{Ὀνασιμίδᾱς},
\textel{Ἐπιχαρίδᾱς Φύλλιος},
\textel{Ἀντίγονον Ἀσκλᾱπιάδᾱο Μακεδόνα},
\textel{Σωσίβιον Διοσκουρίδᾱο Ἀλεξανδρεῖα},
\textel{Ξάνθιππον Κενδήβα Πισίδᾱν},
and \textel{Καφισίας Ἀργικλίδᾱο}.
\ig{23} has the name \textel{Διόδοτον Ἡρακλίδᾱο Κουζικηνώς},
and \ig{531} has the fragmentary name \textel{Διουσκορίδᾱν [— — —]ρω Ἀθανεῖον}.

The singular accusative and dative of \textel{δᾶμος} are also attested in the Tanagran inscriptions.
The accusative \textel{δᾶμον} occurs in \ig{322}.
The dative \textel{δ\A{ᾱ}μοι} occurs in \ig{505 - 508} and \ig{524}.
An alternative dative \textel{δ\A{ᾱ}μυ} occurs in \ig{504}, \ig{509}, \ig{523}, and \ig{531}.
This form is easily identified as a dative since every instance of \textel{τῦ δ\A{ᾱ}μυ} occurs as the indirect object of \textel{δεδόχθη},
as do every instance of \textel{τοῖ δ\A{ᾱ}μοι}.
These inscriptions potentially capture a moment in the middle of a sound change,
where some of the city's population shifted the segment \ortho{\textel{οι}}  to \ortho{\textel{υ}} (whatever their quality).
While the whole population had not accepted the shift yet,
it was common enough as a synchronic variant for the entire population to recognize it quickly.
Furthermore,
while on the topic of declensions,
the singular forms of \textel{γᾶ} are also partly attested.
The singular genitive \textel{γᾶς} occurs in \ig{504 - 509},
\ig{522 - 524},
and \ig{521}. 
The accusative \textel{γᾶν} occurs in \ig{504 - 509},
\ig{522 - 424},
and \ig{531}. 

The root \groot{μᾱτ(ε)ρ} occurs five times,
though they are interesting for a similar reason as \textel{τῦ δ\A{ᾱ}μυ} above.
The form \textel{Δᾱμᾱτρίω} occurs four times in the Tanagran inscriptions.
These are identifiable as genitives equivalent to \textel{Δᾱμᾱτρίου} by their role
in the phrase \textel{μεινὸς/μηνὸς Δᾱμᾱτρίω} ``of the month of Damatrios,''
which is a standard formula in Bœotia for saying that the inscription was made in the fourth month of the year. \autocite[162]{McLean}
\textel{Δᾱμ\A{ᾱ}τριος} here serves as the name of the month in the same way that April could refer to a month or an individual.


The evidence from this century affirms that Tanagra should remain categorized as on the \textel{δᾶμος} side of our isogloss line.
Though the locals were midway through some vowel shift involving diphthongs with the second component \ortho{\textel{ι}},
the inscriptions indicate that Tanagrans retained their long alphas in all positions.
Therefore,
the isogloss line should be drawn along the same path through the mountains as the line from the fourth century \textel{bce}
and should come to a stop at a position south of Tanagra.

\subsection{3\textsuperscript{rd} cen.  Oropus}
The last subset, and the longest, comes from Oropus. 
There are,
in ascending order, 
one instance of the root \groot{δᾱλ},
one of the suffix \groot{-δᾱς},
eleven of the root \groot{μᾱτ(ε)ρ},
sixty-five of the root \groot{βουλ},
and one hundred and twelve of the root \groot{δᾱμ}.

\ig{291} contains the root \groot{δᾱλ} as part of the name \textel{Χοιρύλον Χοιρύλου Δήλιον},
which identifies the individual as a foreigner.
\ig{290} contains the suffix \textel{-δᾱς} in the name \textel{Ὀλουμπίων Ἀρχελαΐδᾱο Ταναγρῆος},
yet another foreigner.

Every instance of \groot{μᾱτ(ε)ρ} also has replaced its long alpha with an eta.
\ig{265} shows the names \textel{Δημήτριος} and \textel{Δημήτριον Ζωΐλου Ὀλύνθιον},
which identifies the individual as a foreigner
(\textel{Ὄλυνθος, ἡ τῆς Χαλκιδῆς πόλις}).
\ig{319 - 321} each contain the name \textel{Δημήτριος Σωφίλου};
and \ig{263}, 
\textel{Σώφιλος}.

The nominative \textel{Μητρόδωρος Ἕρμωνος} is attested in \ig{316}.
The genitive \textel{Μητροδώρου} is attested in \ig{299}.
The last instance,
\textel{Ἡραῖου Μητροφάνου Ἀθηναῖον} in \ig{302} indicates a foreigner.

The root \groot{βουλ} occurs sixty-five times in the Oropian data.
Of these,
eleven are either verb forms or proper nouns:
such instances will be ignored since this root is primarily helpful for its case endings.
That leaves us with fifty-four instances,
all of which are in the dative.
Twenty-four take the expected form \textel{βουλῆι},
but the other thirty read \textel{βουλεῖ}.
All instances occur as the indirect object of the verb \textel{δεδόχθαι} ( $\gets$ \textel{δοκέω}),
and usually as part of the phrase \textel{τῆι βουλῆι καὶ τῶι δάμωι}.
As both spellings point to the same underlying case,
the forms will be treated as equivalent. 
	
The root \groot{δᾱμ} has an even more significant presence in the Oropian data,
appearing in thirty-two proper names and seventy-eight instances of the word \textel{δῆμος} in one case or another,
though all invariably singular.
The names where \textel{Δημ-} is the primary component include seven instances of the name \textel{Δημήτριος} variously in the nominative, genitive, and accusative
as well as the third declension feminine \textel{Δημητρίας}.
Then there are six instances of the term \textel{Δημοκράτου};
three instances of \textel{Δημοστράτης};
and a single instance of \textel{Δημοῦς}.
	
Names where \textel{Δημ-} is not the primary component include two instances of the genitive \textel{Εὐδήμου},
one instance of \textel{Ἐχέδημος},
one instance of the nominative \textel{Κριτόδημος}
and one instance of the accusative \textel{Κριστόδημον},
one instance each of the genitive \textel{Πυρριδήμου} and the nominative \textel{Σθενέδημος},
and three instances of the genitive \textel{Χαριδήμου}.
	
There are three important counterexamples.
First,
\ig{296} contains the name \textel{Δᾱμᾱτρίου}.
Fortunately,
this can be explained away with ease.
The name comes as part of the phrase \textel{μηνὸς Δαματρίου ἕκτηι ἀπίοντος}
``on the sixth day of the waning of the month of Damatrios,''
that is to say,
``on the twenty-fifth day of the fourth month.''
\textel{Δᾱμ\A{ᾱ}τριος} is the common Bœotian name for the fourth month,
so it is not surprising to see it appear in a Bœotian town.
Moreover,
the Oropians were in a league with the other Bœotians at the time;
\ig{296} begins \textel{ἄρχοντος ἐν κοινοῖ Βοιωτῶν Διονυσίου}
``During the Dionysios' archonship in the league of the Bœotians.''
With this in mind,
the presence of a Bœotian calendrical term is hardly remarkable.

	
Next,
\ig{239} has the name \textel{Χαριδ\A{ᾱ}μου},
which can be explained in a similar way to \textel{Δαματρίου}.
The name is part of a formulaic phrase:
the inscription begins with \textel{ἄρχοντος ἐν κοινοῖ Χαριδ\A{ᾱ}μου}
``During Kharidamos' archonship.''
Given Oropus' political relationship with the rest of Bœotia,
it is also unremarkable to see political figures with Bœotian name varieties.
	
Last,
\ig{237} has the name \textel{Ἐργοκλῆς Χαριδ\A{ᾱ}μου}.
He was the monument's dedicant,
as indicated through the formula \textel{ἐπειδὴ Ἐργοκλῆς Χαριδάμου εὔνους ὢν διατελεῖ τεῖ πόλει}
``Inasmuch as Ergokles Kharidamos continues to be generous to the city.''
There is no direct evidence as to why Ergokles' apparently had the non-Oropian name \textel{Χαριδᾶμος},
but the possibilities are plenty.
Ergokles could have been a foreigner,
or a first-generation Oropian,
or the son of a foreign father and an Oropian mother.
The list goes on,
but regardless of all that,
Ergokles' father was simply not statistically significant enough to negate all the other Oropian evidence.
He should have tried putting up his own monument.
	
Taken as a whole,
the evidence from Oropus confirm that the city should be classified as on the \textel{δῆμος} side of the isogloss line.
The inscriptions show a dialect completely sold on the Attic-Ionic vowel shift.
I suspect this is partly due to the realities of travel by foot or by animal power.
If we temporarily forget about sea travel,
any Athenian traveling north had two options.
They may have traveled northwest to Eleusis,
then north through the mountains,
then into the valleys below Tanagra.
This is a great way to get taxed,
or ``taxed,''
if Attic-Bœotian relations were in another cold patch.
Alternatively,
they could have traveled northeast in the direction of Marathon,
then enjoyed a trip along the coast as they followed the Euboic Sea,
which would take them just below the Amphiareion and into Oropus.
I suspect most north-south travel followed a route along these lines.