\section{Conclusion}

The sample tells a surprisingly conservative story about the \textel{δᾶμος/δῆμος} isogloss. 
he fifth century \textel{bce} sets a strong precedent for the Bœotians.
It was established early that Tanagra was a \textel{δᾶμος} city,
though without much local context to help understand that,
and helped to define some of the fundamental suffixes and roots used in the study. 

For the most part,
the fourth century \textel{bce} looks how we would expect.
The average speaker in Megara did not participate in the Attic-Ionic Vowel Shift
and retained the long alpha in all positions.
So,
too,
did the average speaker in Tanagra. Between Megara and Tanagra,
we see the general development of the \textel{δᾶμος/δῆμος} isogloss line,
as it traveled north from the area around Eleusis,
through the mountains,
and into the valleys under Tanagra.
Oropus,
however,
breaks with this trend.
By the fourth century,
the general speaker in Oropus had adopted a significant number of Attic and Ionic features.
Most conspicuous is the shift from a long alpha to an eta,
a specifically Attic-Ionic feature.
Despite Oropus being a Bœotian city,
and Oropia a Bœotian subregion,
the \textel{δᾶμος/δῆμος} isogloss line passes between Oropus and Tanagra,
establishing a dialectal difference between the cities.
The same status is reflected in the fifth century \textsc{bce}.
The increased sample size only confirms the classification of our cities,
especially concerning Oropus,
which saw a significant boost in representation,
as well as Pagæ and Ægosthena,
which help to establish more context for the Doric dialect as spoken in Megaris.