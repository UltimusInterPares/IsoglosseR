\section{Methodlology}

This study works with Greek inscriptions from the fifth to the third century \textsc{bce}.
All inscriptions are taken from volume seven of the Inscriptiones Græcæ (IG VII),
a multi-volume collection of Greek texts,
through a digital copy provided by the Packard Humanities Institute (PHI).
The PHI hosts digitizations ofmultiple epigraphic collections,
and often more than one text will publish the same inscription.
When this happens, the PHI will digitize the text once,
then replace the text in all other books with a link to that digitization.
When an inscription from the sample is cross-listed,
the inscription is taken from the corresponding entry in the alternate book,
but the original IG VII number is maintained.

The study focuses on population centers near the borders of Attica.
Megara, Pagæ, and Ægosthena represent the Dorians,
while Oropus and Tanagra represent the Bœotians.
The Oropian inscriptions are all taken from the same sanctuary,
alled the Amphiareion, 
and are cross-listed in the collection Hoi Epigraphes Tou Oropou (EO).

The first one hundred inscriptions were taken from each location as available
and were collected in the order in which the IG presents them.
The original sample has one hundred entries from Megara (IG VII 1 - 100), 
twenty from Pagæ (IG VII 188-207), 
twenty-eight from Ægosthena (IG VII 207 - 234), 
one hundred from Oropus (IG VII 235 - 334), 
and one hundred from Tanagra (IG VII 504 - 604). 
That gives one hundred and forty-eight Doric inscriptions and two hundred Bœotian inscriptions,
 for a total of two hundred and forty-eight original data. 
 These inscriptions then underwent two reductions based on their date of creation.

First, some data were eliminated for not carrying enough chronological information. 
One hundred and eight inscriptions were undated: 
nineteen from Megara,
thirteen from Pagæ, 
eight from Ægosthena, 
four from Oropus, 
and sixty-four from Tanagra. 
These data were rejected as estimating a date or date range is beyond the scope of this study. 
This elimination has the unfortunate effect of removing such interesting inscriptions as IG VII 586 \textel{ἐπ[ὶ] ϙόραε \{κόραι\}}
and IG VII 593 \textel{ἐπὶ Ϝ{\h}εκαδ\A{ᾱ}μοε \{Ἀκαδήμῳ\} \Sm{ε}μί}.

However, 
it is necessary to eliminate data that would be useless in establishing a century-by-century look at the dialects of Central Greece.
Second, some data were eliminated for not falling within the scope of the study. 
The sample taken from the IG contained twenty-two inscriptions from the second century \textsc{bce}
and fifteen from the first century \textsc{bce}. 
That leaves one hundred and forty inscriptions in the revised sample. 
Of these, fifteen are Megarian, 
two are Pagæan, 
and fourteen are Ægosthenian, 
while seventeen are Tanagran, 
and a whopping eighty-eight are Oropian. 
Doric Greek then accounts for twenty-five percent ($\frac{15+2+18}{140}=0.25$) of the inscriptions, 
while Bœotian Greek accounts for the other seventy-five ($\frac{88+17}{140}=0.75$). 

Across centuries, one inscription comes from the fifth century \textsc{bce} ($\frac{1}{140}=.007$),
twenty-two from the fourth ($\frac{22}{140}=.157$),
and one hundred and seventeen from the third ($\frac{117}{140}=.835$).
The makeup of each century's inscriptions will happen on a century-by-century basis. 

This study was conducted with the assumption that an inscription is indicative of its local majority dialect. 
In an era before digital transfers of information, 
these inscriptions would serve as a necessary component in disseminating information deemed critical to the public
and so are presumed to have been written in a dialect that would be immediately understandable to the majority of people in the area. 

This study also assumes a working knowledge of Attic and Ionic Greek, 
and assumes that most dialectal forms can be converted to Attic or Ionic by the reader.
 Some core roots are glossed for clarity. 
 Some more inscrutable forms are given brief etymological explanations. 
 Some words are translated for clarity if they are obscure, 
 in order to maintain a distinction with similar words, 
 or if they happen to be neat. 
 Long alphas are written with a macron above them unless they are accented with a circumflex, 
 which makes their quantity clear.
